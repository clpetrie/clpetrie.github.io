\documentclass[letterpaper,10pt]{article}

\usepackage{hyperref}
\usepackage[hmargin=1.5cm,vmargin=1.5cm]{geometry}

\def\name{Cody L. Petrie - Curriculum Vitae}

\geometry{
 body={6.5in, 8.5in},
  left=1.0in,
  top=1.0in
}


\usepackage{sectsty}
\sectionfont{\rmfamily\mdseries\Large}
\subsectionfont{\rmfamily\mdseries\itshape\large}

\setlength\parindent{0em}

\renewenvironment{itemize}{
  \begin{list}{}{
    \setlength{\leftmargin}{1.5em}
  }
}{
  \end{list}
}

\begin{document}

{\huge \name}

\vspace{0.25in}

\begin{minipage}{0.45\linewidth}
  \href{http://www.asu.edu/}{Arizona State University} \\
  Department of Physics\\
  PSF 470 \\
  Tempe, AZ 85281
\end{minipage}
\begin{minipage}{0.45\linewidth}
  \begin{tabular}{ll}
    Phone: & 480-392-3214 \\
    Email: & \href{mailto:cody.petrie@asu.edu}{\tt cody.petrie@asu.edu} \\
  \end{tabular}
\end{minipage}

\section*{Education}

\begin{itemize}

  \item $\bullet$ B.S. Physics, Brigham Young University, Aug 2014.

  \item $\bullet$ TALENT Summer School on Nuclear Quantum Monte Carlo Methods, July 2016

\end{itemize}

\section*{Research Experience}
\begin{itemize}

% \item \textbf{Computational Plasma:} September 2013 - August 2014
%   \begin{itemize}
%    \item - I am learning how to use DSMC methods to model particle collisions. This experience will be expanded to be able to model a helium plasma that is being used as an EUV light source. The goal is to be able to increase the intensity of the EUV light that arrives at a reflecting sample.
%    \end{itemize}

 \item \textbf{Quantum Monte Carlo for Nuclear Systems:} August 2014 - Present
  \begin{itemize}
     \item - I am currently using quantum Monte Carlo methods to solve many-body problems in nuclear physics. I have added quadratic spin-isospin dependent correlations to the trial wave function which improves statistics and energy estimates. These additional correlations have the greatest effect on systems of many nucleons.
   \end{itemize}

 \item \textbf{Coupling of Nano Systems with Electromagnetic Fields:} January 2015 - April 2015
  \begin{itemize}
     \item - I used the Finite Difference Time Domain method to calculate the interaction coupling between nano particles such as Ag islands or spheres with surface plasmons on a Si substrate. 
  \end{itemize}

 \item \textbf{HIV Incidence Estimation:} May 2012 - July 2012
  \begin{itemize}
     \item - I computationally estimated HIV incidence based on serological data of diagnosed cases. I used a combination of survey and Bayesian statistics. This research was part of a summer Science Undergraduate Laboratory Internship (SULI) through the DOE during the summer of 2013.
   \end{itemize}

  \item \textbf{Experimental/Computational Extreme Ultraviolet (EUV) Optics:} March 2011 - August 2014
   \begin{itemize}
      \item - I used geometrical optics, physical optics and direct calculations using Maxwell's equations to calculated reflection from thin film surfaces. These calculations were compared to reflection measurements that I took of EUV light from thin film surfaces of varying roughness. Comparing the calculated reflectances from surfaces with various roughnesses to the measured reflectances I was able to estimate the roughness of the thin films.
  \end{itemize}
\end{itemize}

\section*{Grants \& Awards}
\begin{itemize}
  \item $\bullet$ Summer University Graduate Fellowship at ASU, Summer 2015.
  \item $\bullet$ Department of Physics Graduate Fellowship at ASU, Fall 2014.
  \item $\bullet$ Office of Research and Creative Activities Grant at BYU, Academic year of 2013-2014.
\end{itemize}

\section*{Computational Experience}
 \subsection*{Languages}
\begin{itemize}
  \item Fortran \\ Python \\ C++ \\ Matlab \\ Mathematica \\ R
\end{itemize}
\subsection*{Operating Systems}
\begin{itemize}
  \item Linux \\ Windows \\ Mac
\end{itemize}

\section*{Teaching Experience}
\begin{itemize}
  \item \textbf{Teaching Assistant:} General Physics Laboratory (non-calculus based), January 2016-Present, ASU.
  \item \textbf{Teaching Assistant:} University Physics Laboratory 2 (calculus based), August 2015-December 2015, ASU.
  \item \textbf{Teaching Assistant:} University Physics Laboratory 1 (calculus based), August 2014-April 2015, ASU.
  \item \textbf{Physics Tutor:} Tutor for both calculous and non-calculous based classes on waves, optics, thermodynamics, special relativity, and electricity and magnetism, Jan-Apr 2014, BYU.
  \item \textbf{Teaching Assistant:} Introduction to Analog and Digital Circuits, Sep-Dec 2013, BYU.
  \item \textbf{Teaching Assistant:} Classical Mechanics, Sep-Dec 2013, BYU.
  \item \textbf{Teaching Assistant:} Introduction to Waves, Optics, and Thermodymanics (Physics Major Section), Jan-Apr 2013, BYU.
  \item \textbf{Teaching Assistant:} Introduction to Electricity and Magnetism, Sep-Dec 2012, BYU.
\end{itemize}

\section*{Publications}
\begin{enumerate}
   \item Ethan Obie Romero-Severson, \textbf{Cody L. Petrie}, Edward Ionides, Jan Albert, Thomas Leitner. Trends of HIV-1 incidence with credible intervals in Sweden 2002-09 reconstructed using a dynamic model of within-patiend IgG growth. Int. J. Epidemiol., 2015, Vol. 0, No. 0.
   \item  \textbf{Cody L. Petrie}, Joshua Marx, David Squires, R. Steven Turley. Determining thin-film roughness with extreme ultraviolet reflection. J. Utah Acad. Sci. Arts Letts., 92, 239-255 (2015).
   \item \textbf{Cody L. Petrie}, Determining thin film roughness with EUV reflection, Brigham Young University (2014), Senior Thesis.
   \item  Quintin Nethercott, \textbf{Cody L. Petrie}, R. Steven Turley. Non-specular reflection in the extreme ultraviolet. J. Utah Acad. Sci. Arts Letts., 89, 181-193 (2012).
\end{enumerate}

\section*{Talks and Posters}
\begin{enumerate}
  \item ``Determining Thin Film Roughness with Extreme Ultraviolet Refletion," \textbf{Cody L. Petrie}, R. Steven Turley. Utah Academy of Sciences, Arts and Letters, St. George Utah, April 11, 2014.
  \item ``Determining Thin Film Roughness with Extreme Ultraviolet Reflection," \textbf{Cody L. Petrie}. BYU Student Research Conference, Provo Utah, March 15, 2014
  \item ``Using EUV Reflection to Understand Thin Film Surfaces," \textbf{Cody L. Petrie}, R. Steven Turley. Utah Academy of Sciences, Arts and Letters, Orem Utah, April 12, 2013.
  \item ``Using EUV Reflection to Understand Thin Film Surfaces," \textbf{Cody L. Petrie}. BYU Student Research Conference, Provo Utah, March 9, 2013
  \item ``Determining Thin Film Roughness with Extreme Ultraviolet Light," \textbf{Cody L. Petrie}, R. Steven Turley. Annual Meeting of the Four Corners Section of the APS, Socorro New Mexico, October 26, 2012.
  \item ``Nonspecular reflectance in the extreme ultraviolet," Quintin Nethercott, \textbf{Cody L. Petrie}, R. Steven Turley. Utah Academy of Sciences, Arts and Letters, Logan Utah, April 13, 2012.
  \item ``Improving thin film thickness uniformity," Jordan Bell, \textbf{Cody L. Petrie}. BYU Student Research Conference, Provo Utah, March 12, 2012
\end{enumerate}

\end{document}
